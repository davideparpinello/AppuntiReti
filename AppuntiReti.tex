\documentclass{report}
\usepackage[utf8]{inputenc}
\usepackage[italian]{babel}
\usepackage[linktoc=all]{hyperref}
\usepackage{graphicx}
\usepackage{mathtools}
\usepackage{float}
\hypersetup{
	colorlinks,
	citecolor=black,
	filecolor=black,
	linkcolor=black,
	urlcolor=black
}
\usepackage[a4paper,includeheadfoot,margin=2.54cm]{geometry}
\usepackage{fancyhdr}
\newcommand\hr{\par\vspace{-.5\ht\strutbox}\noindent\hrulefill\par}
\usepackage{enumitem}


\begin{document}
	
	\author{Davide Parpinello}
	\title{%
			\begin{Large}
				Appunti di\\
			\end{Large}
		Reti (F. Granelli)}
	\date{Aprile 2020}
	\maketitle
	
	\tableofcontents
	\renewcommand{\chaptermark}[1]{%
		\markboth{#1}{}}
	\addtocontents{toc}{\protect\hypertarget{toc}{}}
	\pagestyle{fancy}
	\fancyhf{}
	\rhead{\hyperlink{toc}{Reti}}
	\lhead{\leftmark}
	\rfoot{\thepage}
	
	\chapter{Roadmap}
	\section{Internet}
	
	Internet è costituito da milioni di dispositivi, chiamati \textbf{sistemi terminali}, collegati tra loro da collegamenti in rame, fibre ottiche, oppure via radio come onde elettromagnetiche o satelliti.
	\medskip\\
	La frequenza di trasmissione in internet è data dall'ampiezza di banda disponibile.
	\medskip\\Sulla rete internet inoltre sono presenti particolari host, denominati \textbf{router}, che si occupano di instradare i pacchetti verso la loro destinazione finale.
	\medskip\\Nello scambio di messaggi tra host vengono implementati dei \textbf{protocolli}, che ne definiscono formato e ordine d'invio, così come le azioni intraprese in fase di trasmissione e/o ricezione di un messaggio o un altro evento.
	\medskip\\Internet è un'infrastruttura di comunicazione per applicazione distribuita, viene anche chiamato "rete delle reti" ed è organizzato in modo gerarchico.
	\section{Ai confini della rete}
	Sul bordo della rete internet troviamo applicazioni e sistemi terminali, raggruppati tra loro e connessi tra di loro mediante collegamenti cablati e wireless.
	\medskip\\I sistemi terminali (o host) fanno girare diversi programmi applicativi e possono essere organizzati con architettura client/server oppure P2P.
	\medskip\\L'accesso a internet può avvenire mediante diversi modi:
	\begin{itemize}
		\item \textbf{Accesso residenziale:} viene utilizzato un modem dial-up o DSL.
		\item \textbf{Accesso aziendale:} una LAN collega i sistemi terminali di aziende e università all'\textbf{edge router}, i sistemi terminali sono collegati tra loro mediante uno switch ethernet.
		\item \textbf{Accesso wireless:} i terminali vengono collegati mediante \textbf{access point}.
		\item \textbf{Reti domestiche:} sono costituite da un modem DSL o via cavo, un router/firewall/NAT, switch ethernet e accesso wireless. Spesso queste funzioni vengono raggruppate in un unico dispositivo (modem/router).
	\end{itemize}
	\section{Il nucleo della rete}
	Al centro della rete si trovano invece router interconnessi tra loro, che creano quindi una rete delle reti.
	\medskip\\I dati nel nucleo della rete vengono trasferiti con due modalità differenti:
	\begin{itemize}
		\item \textbf{Commutazione di circuito:} è presente un circuito dedicato per l'intera durata della sessione
		\item \textbf{Commutazione di pacchetto:} i messaggi di una sessione utilizzano le risorse su richiesta, di conseguenza potrebbero dover attendere per accedere a un collegamento.
	\end{itemize}
	\subsection{Esempio di commutazione di circuito}
	Consideriamo un file L di 640.000 bit, un bitrate totale C da 1.536 Mbps, TDM con 24 slot/s, 500ms per stabilire la connessione.
	\medskip\\ Trovo inizialmente la capacità di un singolo slot:\[C_{1 slot} = \frac{C}{24} = 0.064 Mbps = 64 Kbps\]
	Successivamente calcolo il tempo necessario alla trasmissione:\[T_{tx} = \frac{L}{C_{1 slot}} = 10s\]
	Infine, calcolo il tempo totale:\[T_{tot}=500ms+10s=10,5s\]
	\subsection{Esempio di commutazione di circuito}
	I secondi necessari per trasmettere un pacchetto in uscita su un collegamento da R bps sono dati da L/R, mentre il ritardo 3L/R
	\subsection{Confronto fra commutazione di pacchetto e di circuito}
	La commutazione di pacchetto consente un utilizzo della rete da parte di maggiori utenti, ed è ottima per i dati a raffica.\medskip\\ Dal lato negativo, presenta un'eccessiva congestione, causando ritardi e perdite di pacchetti. Sono quindi necessari protocolli per il trasferimento affidabile e per il controllo della congestione.
	\subsection{Struttura gerarchica}
	La rete internet ha una struttura fondamentalmente gerarchica:
	\begin{itemize}
		\item Al centro sono presenti \textbf{ISP di livello 1}, che offrono una copertura nazionale e/o internazionale
		\begin{itemize}
			\item Comunicano tra loro come fossero pari
		\end{itemize}
		\item \textbf{ISP di livello 2:} ISP più piccoli, copertura nazionale/distrettuale.
		\begin{itemize}
			\item Si può connettere solo ad alcuni ISP di livello 1 e ad altri di livello 2
			\item Paga l'ISP di livello 1 che gli fornisce la connettività per il resto della rete
		\end{itemize}
		\item \textbf{ISP di livello 3 e ISP locali (di accesso):}
		\begin{itemize} 
			\item sono le reti "ultimo salto", le più vicine agli host
			\item Sono clienti degli ISP di livello superiore che li collegano all'intera internet.
		\end{itemize}
	\end{itemize}
	Un pacchetto attraversa un sacco di reti, dal livello più basso fino al principale e poi nuovamente a scendere.
	
	
\end{document}