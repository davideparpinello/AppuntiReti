% Options for packages loaded elsewhere
\PassOptionsToPackage{unicode}{hyperref}
\PassOptionsToPackage{hyphens}{url}
%
\documentclass[
]{article}
\usepackage{lmodern}
\usepackage{amssymb,amsmath}
\usepackage{ifxetex,ifluatex}
\ifnum 0\ifxetex 1\fi\ifluatex 1\fi=0 % if pdftex
  \usepackage[T1]{fontenc}
  \usepackage[utf8]{inputenc}
  \usepackage{textcomp} % provide euro and other symbols
\else % if luatex or xetex
  \usepackage{unicode-math}
  \defaultfontfeatures{Scale=MatchLowercase}
  \defaultfontfeatures[\rmfamily]{Ligatures=TeX,Scale=1}
\fi
% Use upquote if available, for straight quotes in verbatim environments
\IfFileExists{upquote.sty}{\usepackage{upquote}}{}
\IfFileExists{microtype.sty}{% use microtype if available
  \usepackage[]{microtype}
  \UseMicrotypeSet[protrusion]{basicmath} % disable protrusion for tt fonts
}{}
\makeatletter
\@ifundefined{KOMAClassName}{% if non-KOMA class
  \IfFileExists{parskip.sty}{%
    \usepackage{parskip}
  }{% else
    \setlength{\parindent}{0pt}
    \setlength{\parskip}{6pt plus 2pt minus 1pt}}
}{% if KOMA class
  \KOMAoptions{parskip=half}}
\makeatother
\usepackage{xcolor}
\IfFileExists{xurl.sty}{\usepackage{xurl}}{} % add URL line breaks if available
\IfFileExists{bookmark.sty}{\usepackage{bookmark}}{\usepackage{hyperref}}
\hypersetup{
  hidelinks,
  pdfcreator={LaTeX via pandoc}}
\urlstyle{same} % disable monospaced font for URLs
\usepackage{graphicx}
\makeatletter
\def\maxwidth{\ifdim\Gin@nat@width>\linewidth\linewidth\else\Gin@nat@width\fi}
\def\maxheight{\ifdim\Gin@nat@height>\textheight\textheight\else\Gin@nat@height\fi}
\makeatother
% Scale images if necessary, so that they will not overflow the page
% margins by default, and it is still possible to overwrite the defaults
% using explicit options in \includegraphics[width, height, ...]{}
\setkeys{Gin}{width=\maxwidth,height=\maxheight,keepaspectratio}
% Set default figure placement to htbp
\makeatletter
\def\fps@figure{htbp}
\makeatother
\setlength{\emergencystretch}{3em} % prevent overfull lines
\providecommand{\tightlist}{%
  \setlength{\itemsep}{0pt}\setlength{\parskip}{0pt}}
\setcounter{secnumdepth}{-\maxdimen} % remove section numbering

\author{}
\date{}

\begin{document}

\hypertarget{header-n0}{%
\section{Livello di collegamento: introduzione e
servizi}\label{header-n0}}

A livello di collegamento definiamo come \textbf{nodi} host e router,
che vengono collegati tra loro da \textbf{collegamenti (link)} di
tipologia cablata, wireless o LAN. Le unità di dati scambiate a livello
link sono chiamate \textbf{frame}.

In breve, i protocolli a livello di collegamento si occupano del
trasporto di datagram lungo un singolo canale di comunicazione. Questi
protocolli possono essere differenti sui vari collegamenti che seguirà
il datagram e i servizi erogati possono essere differenti: non tutti i
protocolli, ad esempio, forniscono consegna affidabile.

\hypertarget{header-n4}{%
\subsection{Servizi offerti a livello di link}\label{header-n4}}

\begin{itemize}
\item
  \textbf{Framing}

  \begin{itemize}
  \item
    I protocolli incapsulano i datagram del livello di rete in frame a
    livello di link
  \item
    Il protocollo MAC controlla l'accesso al mezzo trasmissivo (il
    collegamento)
  \item
    Viene utilizzato l'indirizzo MAC (diverso dall'IP) per identificare
    i nodi di origine e destinazione
  \end{itemize}
\item
  \textbf{Consegna affidabile}

  \begin{itemize}
  \item
    Non necessaria sui collegamenti a basso numero di errori sui bit
    (fibra ottica, cavo coassiale, doppino intrecciato)
  \item
    Utilizzata nei collegamenti soggetti a elevato tasso di errori
    (wireless)
  \end{itemize}
\item
  \textbf{Controllo di flusso:} non saturare il nodo ricevente
\item
  \textbf{Rilevazione degli errori:} causati dall'attenuazione del
  segnale e dal rumore elettromagnetico, il ricevente individua gli
  errori grazie a un bit di controllo inserito nel frame
\item
  \textbf{Correzione degli errori:} il ricevente determina l'errore e lo
  corregge
\item
  \textbf{Half-duplex e full-duplex}
\end{itemize}

\hypertarget{header-n30}{%
\subsection{Dov'è implementato il livello link}\label{header-n30}}

Viene implementato in tutti gli host grazie ad un adattatore (o
\textbf{NIC}, network interface card) che implementa il livello di
collegamento e fisico ed è una combinazione di hardware, software e
firmware.

L'adattatore ha il ruolo in trasmissione di incapsulare i datagram nei
frame impostando bit di controllo errori, trasferimento affidabile,
controllo di flusso, ecc e dall'altro lato di individuare errori ed
estrarre i datagram passandoli al nodo.

\hypertarget{header-n33}{%
\section{Tecniche di rilevazione e correzione degli
errori}\label{header-n33}}

La rilevazione degli errori non è attendibile al 100\%, infatti è
possibile che ci siano errori non rilevati e per ridurre che questo
accada le tecniche più sofisticate prevedono un'elevata ridondanza.

\hypertarget{header-n35}{%
\subsection{Controllo di parità}\label{header-n35}}

Nel caso di \textbf{unico bit di parità} è presente un solo bit che
consente di riconoscere che si è verificato almeno un errore in un bit.

Nel caso della \textbf{parità bidimensionale} è possibile individuare e
correggere il bit alterato.

\emph{figura 5-12 bidimensionale}

\hypertarget{header-n39}{%
\section{Protocolli di accesso multiplo}\label{header-n39}}

Esistono due tipi di collegamenti di rete:

\begin{itemize}
\item
  collegamento \textbf{punto-punto} (PPP), utilizzato per connessioni
  telefoniche o collegamenti punto-punto tra Ethernet e host
\item
  collegamento \textbf{broadcast} (cavo o canale condiviso) come
  l'Ethernet tradizionale o wireless
\end{itemize}

Nel caso di una connessione a un canale broadcast condiviso è consentita
la connessione anche a migliaia di nodi: si genera quindi una
\textbf{collisione} quando i nodi ricevono più di un frame
contemporaneamente.

I protocolli ad accesso multiplo consentono quindi di definire le
modalità con cui i nodi regolano le loro trasmissioni sul canale, la cui
comunicazione utilizza il canale stesso (non è presente un canale fuori
banda).

Il protocollo di accesso multiplo ideale sarebbe un protocollo
decentralizzato e semplice che suddivide il tasso trasmissivo tra il
numero di nodi che devono inviare dati. Ovviamente tale protocollo non è
possibile

I protocolli di accesso multiplo esistenti vengono classificati come
segue:

\begin{itemize}
\item
  Protocolli a \textbf{suddivisione del canale}: il canale è suddiviso
  in parti (slot di tempo, frequenza, codice) ed allocate a uno
  specifico nodo per utilizzo esclusivo
\item
  Protocolli ad \textbf{accesso casuale}: nessuna divisione, si possono
  verificare collisioni, i nodi ritrasmettono ripetutamente i pacchetti
\item
  Protocolli a \textbf{rotazione}: ogni nodo ha il suo turno di
  trasmissione, ma quelli che hanno molto da trasmettere potrebbero
  avere turni più lunghi
\end{itemize}

\hypertarget{header-n57}{%
\subsection{Protocolli a suddivisione del canale}\label{header-n57}}

\hypertarget{header-n58}{%
\subsubsection{TDMA: accesso multiplo a divisione di
tempo}\label{header-n58}}

Il protocollo TDMA consiste in turni per accedere al canale,
suddividendolo in intervalli di tempo. Gli slot non usati rimangono
inattivi.

\emph{figura 5-18}

\hypertarget{header-n61}{%
\subsubsection{FDMA: accesso multiplo a divisione di
frequenza}\label{header-n61}}

Il protocollo FDMA suddivide il canale in bande di frequenza e ad ogni
nodo viene assegnata una banda di frequenza prefissata.

\emph{figura 5-19}

\hypertarget{header-n64}{%
\subsection{Protocolli ad accesso casuale}\label{header-n64}}

Nei protocolli ad accesso casuale ogni nodo che deve inviare dati
trasmette alla massima velocità consentita dal canale, senza
coordinazione tra i nodi. Se più nodi stanno trasmettendo si verifica
una collisione. Il protocollo definisce quindi come rilevare le
collisioni e come ritrasmettere in caso di avvenuta collisione.

Alcuni protocolli ad accesso casuale sono slotted ALOHA, ALOHA, CSMA,
CSMA/CD, CSMA/CA.

\hypertarget{header-n67}{%
\subsubsection{Slotted ALOHA}\label{header-n67}}

Nel protocollo slotted ALOHA si assume che tutti i pacchetti abbiano la
stessa dimensione e il tempo sia suddiviso in slot, equivalenti al tempo
di trasmissione di un singolo pacchetto.

Quando un nodo deve spedire, esso attenderà fino all'inizio dello slot
successivo. Se nel mentre non si verifica una collisione il nodo potrà
trasmettere il pacchetto nello slot successivo, altrimenti se avviene la
collisione, essa verrà rilevata prima della fine dello slot e
ritrasmetterà il pacchetto con probabilità \emph{p} durante gli slot
successivi.

Questo protocollo consente ai nodi di trasmettere continuamente alla
massima velocità decidendo indipendentemente quando ritrasmettere
(decentralizzazione). Dall'altro lato alcuni slot presenteranno
collisioni andando sprecati ed altri rimarranno vuoti (inattivi)

\hypertarget{header-n71}{%
\subsubsection{Efficienza di Slotted ALOHA}\label{header-n71}}

\textbf{Efficienza} definita come la frazione di slot vincenti in
presenza di un elevato numero di nodi attivi. Nel caso migliore solo il
37\% degli slot compie lavoro utile.

\hypertarget{header-n73}{%
\subsubsection{ALOHA Puro}\label{header-n73}}

Aloha puro è più semplice e non sincronizzato: quando arriva il primo
pacchetto lo trasmette immediatamente e integralmente nel canale
broadcast. Ci sono però elevate probabilità di collisione (i pacchetti
si sovrappongono tra loro).

L'efficienza di ALOHA puro (18\%) è peggio dello slotted.

\hypertarget{header-n76}{%
\subsubsection{Accesso multiplo a rilevazione della portante
(CSMA)}\label{header-n76}}

CSMA si pone in ascolto prima di trasmettere: se il canale è libero
trasmette l'intero paccheto, se sta già trasmettendo aspetta un altro
intervallo di tempo.

Possono ancora verificarsi collisioni: il ritardo di propagazione fa sì
che due nodi non rilevino la reciproca trasmissione. Se avviene una
collisione, non appena rilevata il nodo cessa immediatamente la
trasmissione. La distanza e il ritardo di propagazione sono fondamentali
per calcolare la probabilità di collisione.

\hypertarget{header-n79}{%
\paragraph{CSMA/CD (collision detection)}\label{header-n79}}

Rilevamento della portante differito, come in CSMA: rileva la collisione
in poco tempo e annulla la trasmissione non appena si accorge che c'è
un'altra trasmissione in corso. La collisione è di facile rilevazione
nelle LAN cablate e difficile nelle LAN wireless.

\hypertarget{header-n81}{%
\subsection{Protocolli MAC a rotazione}\label{header-n81}}

Prendono il meglio dai protocolli precedenti, cercando di ereditare
dalla suddivisione di canale la condivisione equa del canale evitando
congestione, mentre dai protocolli ad accesso casuale ereditano
l'efficacia con carichi non elevati.

Si basano sul protocollo \textbf{polling}: un nodo principale sonda a
turno gli altri. Vengono eliminate collisioni e slot vuoti, si introduce
però il ritardo di polling e il problema che se il nodo master si guasta
il canale resta inattivo.

\hypertarget{header-n84}{%
\subsubsection{Protocollo token-passing}\label{header-n84}}

Un messaggio di controllo circola fra i nodi con un ordine prefissato e
chi ne è in possesso può trasmettere. Questo protocollo è
decentralizzato, altamente efficiente ma il guasto di un nodo può
mettere fuori uso l'intero canale.

\hypertarget{header-n86}{%
\subsection{Riepilogo dei protocolli}\label{header-n86}}

Cosa si può fare con un canale condiviso?

\begin{itemize}
\item
  \textbf{Suddivisione del canale:} per tempo, frequenza o codice (TDM,
  FDM)
\item
  \textbf{Suddivisione casuale} o dinamica:

  \begin{itemize}
  \item
    ALOHA, S-ALOHA, CSMA, CSMA/CD (Ethernet)
  \item
    Rilevamento della portante: facile per tecnologie cablate, difficile
    con wireless
  \item
    802.11 utilizza la variante CSMA/CA
  \end{itemize}
\item
  \textbf{A rotazione}

  \begin{itemize}
  \item
    Polling di un nodo principale o passaggio di un token
  \item
    Completa decentralizzazione ed elevata efficienza
  \item
    Usati in Bluetooth, FDDI, IBM Token Ring
  \end{itemize}
\end{itemize}

\hypertarget{header-n109}{%
\section{Indirizzi a livello di collegamento}\label{header-n109}}

\hypertarget{header-n110}{%
\subsection{Indirizzi MAC e ARP}\label{header-n110}}

L'indirizzo MAC (o fisico o Ethernet), è analogo al numero di codice
fiscale di una persona: ha una struttura piatta e non dipende dalla rete
in cui si è collegati. Dipende dal produttore della scheda di rete e
solitamente ha 48 bit. Viene scritto in esadecimale usando 6 coppie di
cifre esadecimali. L'indirizzo broadcast di livello 2 ha tutti 1
(FF-FF-FF-FF-FF-FF).

\emph{figura 5.36}

Sono gestiti dalla \textbf{IEEE} che vende alle società costruttrici di
adattatori i blocchi di spazio di indirizzi. La società dovrà garantire
l'unicità degli indirizzi.

Il vantaggio dell'orizzontalità del MAC è la portabilità delle schede da
una rete all'altra (cambierà solo l'IP)

\hypertarget{header-n115}{%
\subsubsection{Protocollo per la risoluzione degli indirizzi
(ARP)}\label{header-n115}}

Ogni nodo IP nella LAN ha una tabella ARP, la quale contiene la
corrispondenza tra indirizzi IP e MAC.

\begin{verbatim}
<Indirizzo IP, Indirizzo MAC, TTL>
\end{verbatim}

Il TTL indica quando bisognerà eliminare una voce nella tabella
(tipicamente 20 minuti).

\hypertarget{header-n119}{%
\paragraph{ARP nella stessa sottorete}\label{header-n119}}

ARP è plug-and-play: la tabella si costruisce automaticamente, non è
necessario l'intervento dell'amministratore di rete. Un nodo trasmette
in broadcast il messaggio di richiesta ARP, richiedendo l'indirizzo di
un secondo nodo. Quando quest'ultimo riceve il pacchetto ARP risponderà
comunicando il proprio indirizzo MAC.

\hypertarget{header-n121}{%
\paragraph{Invio verso un nodo esterno alla
sottorete}\label{header-n121}}

Se si vuole inviare un pacchetto da A a B attraverso un router R, il
router stesso avrà due tabelle ARP, una per ciascuna rete IP (LAN) a cui
è collegato.

\begin{enumerate}
\def\labelenumi{\arabic{enumi}.}
\item
  Il nodo A controlla qual'è la destinazione del suo datagram, se è
  nella sua stessa sottorete
\item
  A deve inoltrare il datagram al router R, deve quindi trovare il MAC
  dell'interfaccia del router utilizzando ARP
\item
  A quindi incapsulerà il datagram in un frame di livello 2 indirizzato
  a R, il quale toglierà l'header di livello 2, leggerà l'IP di
  destinazione e cercherà il percorso nella sua tabella di routing.
\item
  Il router R dovrà poi cercare l'indirizzo MAC del destinatario sulla
  seconda sottorete sempre con ARP
\item
  Infine il router R incapsulerà a sua volta il datagram in un frame di
  livello 2 inviandolo al destinatario.
\end{enumerate}

\emph{figura 5-41}

\hypertarget{header-n135}{%
\section{Ethernet}\label{header-n135}}

Ethernet detiene una posizione dominante nel mercato delle LAN cablate:

\begin{itemize}
\item
  è stata la prima LAN ad alta velocità con vasta diffusione
\item
  Più semplice e meno costosa di token ring, FDDI e ATM
\item
  Al passo coi tempi con il tasso trasmissivo: da 10 Mbps finoa 10 Gbps
\end{itemize}

Il progetto originale di Ethernet fù ideato da Bob Metcalfe.

\hypertarget{header-n145}{%
\subsection{Topologia}\label{header-n145}}

La topologia originale era quella a Bus con cavo coassiale, diffusa fino
alla metà degli anni 90.

Le reti odierne seguono la topologia a stella, ogni nodo è collegato a
un hub o commutatore (\emph{switch}) permettendo di eseguire in ogni
nodo un protocollo Ethernet separato non entrando in collisione con
altri.

\emph{figura 5.44}

\hypertarget{header-n149}{%
\subsection{Struttura dei pacchetti Ethernet}\label{header-n149}}

Preambolo, MAC di destinazione, MAC di origine, tipo, payload dati e CRC
(correzione degli errori).

\begin{itemize}
\item
  Il \textbf{preambolo} in specifico è costituito da 8 byte, i primi
  sette sono 10101010 e l'ultimo 10101011, e viene utilizzato per
  "attivare" il ricevitore quando viene inviato un pacchetto e per
  sincronizzare l'orologio con quello del trasmittente.
\item
  Gli \textbf{indirizzi} sono costituiti da 6 byte. Se è presente
  l'indirizzo di destinazione o il broadcast il payload viene trasferito
  direttamente al livello di rete altrimenti il pacchetto viene
  ignorato.
\item
  Il campo \textbf{tipo} consente a Ethernet di supportare i vari
  protocolli di rete (multiplexing).
\item
  Il controllo \textbf{CRC} consente all'adattatore ricevente di
  rilevare la presenza di un errore nei bit del pacchetto.
\end{itemize}

\emph{figura 5.45 pacchetto}

\hypertarget{header-n161}{%
\subsection{Servizio senza connessione non
affidabile}\label{header-n161}}

\begin{itemize}
\item
  \textbf{Senza connessione}: non è prevista nessuna forma di handshake
  preventiva prima di inviare un pacchetto.
\item
  \textbf{Non affidabile}: non esiste riscontro, il flusso dei datagram
  non è garantito poiché il compito viene delegato ai protocolli di
  livello superiore.
\end{itemize}

\hypertarget{header-n167}{%
\subsection{Fasi operative del protocollo CSMA/CD}\label{header-n167}}

\begin{enumerate}
\def\labelenumi{\arabic{enumi}.}
\item
  L'adattatore che riceve un datagram da livello 3 prepara il frame e
  ascolta il canale.
\item
  Se è inattivo inizia la trasmissione, altrimenti resta in attesa.
\item
  Durante la trasmissione, verifica se ci sono altri segnali provenienti
  da altri host.
\item
  Se rileva altri segnali interrompe la trasmissione e invia un segnale
  di disturbo (\emph{jam}) per avvisare della collisione.
\item
  L'adattatore si mette in attesa e viene definito uno slot di attesa
  pari a 512 bit; se viene messo in attesa successivamente l'intervallo
  raddoppia ogni volta. Dopo 10 volte raggiungerà il valore massimo,
  quando scade l'attesa se il canale è inattivo si potrà trasmettere
  nuovamente.
\end{enumerate}

Il segnale \emph{jam} viene trasmesso a un voltaggio più elevato ed è
lungo 48 bit. L'attesa è esponenziale con l'obiettivo di stimare quanti
sono i nodi in attesa coinvolti.

\hypertarget{header-n180}{%
\subsection{Efficienza di Ethernet}\label{header-n180}}

\begin{itemize}
\item
  Tempo di propagazione: tempo massimo che occorre al segnale per
  propagarsi tra due host
\item
  Tempo di trasmissione: tempo necessario per trasmettere un pacchetto
  della maggior dimensione possibile
\end{itemize}

\[efficienza = \frac{1}{1+5t_{prop}/t_{trasm}}\]

\begin{itemize}
\item
  Se il tempo di propagazione tende a 0, l'efficienza tenderà a 1
  (100\%, efficienza massima)
\item
  Per aumentare l'efficienza, in alternativa, si può aumentare il tempo
  di trasmissione.
\item
  Molto meglio di ALOHA: decentralizzato, semplice e poco costoso.
\end{itemize}

\hypertarget{header-n194}{%
\subsection{Ethernet 802.3}\label{header-n194}}

Sono presenti diversi standard Ethernet: il MAC e il frame sono
solitamente standard. Abbiamo però differenti velocità e differenti
mezzi trasmissivi (fibra o cavo).

\emph{figura 5-51}

\hypertarget{header-n197}{%
\subsection{Codifica Manchester}\label{header-n197}}

\emph{figura 5-52}

La codifica Manchester veniva utilizzata in 10BaseT: durante la
ricezione di ciascun bit si verifica una transizione, permettendo di
sincronizzare gli orologi di trasmittenti e riceventi. L'operazione
veniva effettuata a livello fisico.

\hypertarget{header-n200}{%
\section{Switch a livello di collegamento}\label{header-n200}}

\hypertarget{header-n201}{%
\subsection{Hub}\label{header-n201}}

L'hub è un dispositivo "stupido" che opera sui singoli bit:

\begin{itemize}
\item
  riproduce un bit incrementandone l'energia trasmettendolo su tutte le
  interfacce, anche se su alcune c'è un segnale (avverrà una collisione)
\item
  non implementa rilevazione di portante né CSMA/CD
\end{itemize}

\hypertarget{header-n208}{%
\subsection{Switch}\label{header-n208}}

Lo switch è un dispositivo a livello di link, è più intelligente di un
hub e svolge un ruolo attivo:

\begin{itemize}
\item
  filtra e inoltra i pacchetti
\item
  ha una tabella e sa a quale porta inoltrare un pacchetto
\item
  è trasparente agli host
\item
  è un componente plug-and-play con autoapprendimento, non richiede
  l'intervento salvo configurazioni particolari
\end{itemize}

Lo switch consente più trasmissioni simultanee: i pacchetti vengono
infatti bufferizzati e il protocollo Ethernet viene implementato su
ciascun collegamento in entrata, evitando collisioni ma consentendo il
full-duplex. Grazie allo switching avvengono quindi più trasmissioni
simultaneamente.

\hypertarget{header-n220}{%
\subsubsection{Tabella di commutazione}\label{header-n220}}

Componente software utilizzato dallo switch per sapere dove inoltrare i
pacchetti ricevuti. Contiene il MAC del nodo di destinazione e
l'interfaccia associata, assomiglia alle tabelle di instradamento con la
differenza che può essere generata automaticamente dallo switch.

\hypertarget{header-n222}{%
\paragraph{Autoapprendimento}\label{header-n222}}

Quando riceve un pacchetto, lo switch impara l'indirizzo del mittente:
registrerà quindi nella tabella di switching la coppia indirizzo/porta.

\hypertarget{header-n224}{%
\subsubsection{Collegare gli switch}\label{header-n224}}

Gli switch possono essere interconnessi tra loro: gli host avranno la
sensazione di essere sulla stessa rete di livello 2 ma saranno su
differenti domini di collisione.

\hypertarget{header-n226}{%
\subsubsection{Esempio di rete di un'istituzione}\label{header-n226}}

\emph{figura 5-63}

\hypertarget{header-n228}{%
\subsubsection{Switch e router a confronto}\label{header-n228}}

\begin{itemize}
\item
  Entrambi i dispositivi sono store-and-forward

  \begin{itemize}
  \item
    Il router lavora a livello di rete (dominio di broadcast)
  \item
    Lo switch lavora a livello di collegamento (dominio di collisione)
  \end{itemize}
\item
  I router mantengono tabelle d'inoltro e implementano algoritmi
  d'instradamento
\item
  Gli switch mantengono tabelle di commutazione e implementano il
  filtraggio e algoritmi di autoapprendimento
\end{itemize}

\hypertarget{header-n241}{%
\section{PPP: protocollo punto-punto}\label{header-n241}}

Protocollo estremamente semplice, prevede un solo collegamento tra
mittente e destinatario. Non è necessario un protocollo di accesso al
mezzo, non occorre un indirizzo MAC esplicito, veniva usato soprattutto
nei collegamenti ISDN.

Sono presenti protocolli più complessi come \textbf{HDLC} (High-level
Data Link Control).

\hypertarget{header-n244}{%
\subsection{Requisiti di IETF (RFC 1547)}\label{header-n244}}

\begin{itemize}
\item
  \textbf{Framing dei pacchetti:} incapsulare il pacchetto di rete
  dentro una struttura riconoscibile a livello di link e a livello
  fisico
\item
  \textbf{Trasparenza: }non porre nessuna restrizione alla
  configurazione dei dati
\item
  \textbf{Rilevazione di errori}
\item
  \textbf{Disponibilità della connessione: }rilevare eventuali guasti
\item
  \textbf{Negoziazione degli indirizzi di rete: }necessario un
  protocollo per interagire con la rete per ottenere ad esempio un
  indirizzo IP
\end{itemize}

Le altre funzioni come correzione degli errori, controllo di flusso,
riordinamento dei pacchetti \textbf{sono delegati ai livelli superiori.}

\hypertarget{header-n257}{%
\subsection{Formato dei pacchetti dati PPP}\label{header-n257}}

\begin{itemize}
\item
  \textbf{Flag: }ogni pacchetto inizia e termina con un byte di valore
  \textbf{01111110}
\item
  \textbf{Indirizzo: }unico valore \textbf{11111111} broadcast (solo due
  entità che comunicano)
\item
  \textbf{Controllo: }unico valore, tutti i frame sono dello stesso tipo
\item
  \textbf{Protocollo: }qual è il protocollo di livello superiore cui
  appartengono i dati incapsulati
\end{itemize}

\emph{figura 5-70}

\hypertarget{header-n268}{%
\subsubsection{Riempimento dei byte}\label{header-n268}}

Per il requisito di trasparenza non dev'essere possibile inserire nel
campo informazioni la stringa flag \textbf{01111110}, poiché non sarebbe
riconoscibile la fine del frame PPP.

Per ovviare a questo problema si ricorre alla tecnica del byte stuffing:
si aggiunge un byte di controllo pari al flag prima di ogni byte di
dati. In questo modo il destinatario riconoscerà la presenza di dati se
sono presenti due byte \textbf{01111110} consecutivi.

\emph{figura 5-72}

\hypertarget{header-n272}{%
\section{Canali virtuali: una rete come un livello di
link}\label{header-n272}}

\hypertarget{header-n273}{%
\subsection{Il concetto di virtualizzazione}\label{header-n273}}

Per \textbf{virtualizzazione} si intende il processo di sostituzione di
una versione fisica con una rappresentazione software. Consente di
spezzare la dipendenza tra hardware e funzionalità software e una veloce
innovazione, è più facile testare e progettare i servizi senza il
bisogno dell'ambiente fisico. La virtualizzazione consente di isolare e
partizionare le risorse disponibili, oltre alla possibilità di maggior
controllo.

\hypertarget{header-n275}{%
\subsubsection{Tipi di virtualizzazione}\label{header-n275}}

\begin{itemize}
\item
  \textbf{Virtualizzazione hardware:} astrarre la funzionalità logica
  dall'infrastruttura fisica (ad esempio programmazione con compilatore)
\item
  \textbf{Virtualizzazione di rete: }reti virtuali basate su dispositivi
  virtuali, successivamente mappate su risorse fisiche
\item
  \textbf{Cloud: }gestire al meglio grosse quantità di CPU, storage e
  rete fornendo un pool di risorse aggregate, con lo scopo di fornire le
  risorse in maniera scalabile (illusione di risorse infinite)
\end{itemize}

\hypertarget{header-n283}{%
\subsubsection{Storia della virtualizzazione}\label{header-n283}}

La virtualizzazione inizia nell'era nei mainframe negli anni '60 dove le
risorse di computazione venivano condivise tra molti utenti.
Successivamente la virtualizzazione venne utilizzata nei datacenter,
consentendo la suddivisione delle risorse disponibili, arrivando al
giorno d'oggi con la virtualizzazione delle reti.

\hypertarget{header-n285}{%
\subsection{Internet: virtualizzazione delle reti}\label{header-n285}}

Grazie all'indirizzamento IP e ai gateway divenne possibile collegare
tra loro reti diverse per configurazione tra loro che prima non potevano
comunicare. IP consente quindi di virtualizzare le reti.

\hypertarget{header-n287}{%
\subsubsection{Architettura di Cerf e Kahn}\label{header-n287}}

Propone un primo tipo di virtualizzazione, dove il livello 3 (IP) rende
tutto omogeneo, infatti il livello è necessariamente standard. La
virtualizzazione ha consentito di spezzare l'indirizzamento a livello
globale (livello 3) da quello locale (livello 2), virtualizzando le
tecnologie di livello 2 (cavo, satellite, 56K).

\hypertarget{header-n289}{%
\subsubsection{Ethernet e domini}\label{header-n289}}

\begin{itemize}
\item
  \textbf{Dominio di "collisione": }determinato dall'insieme degli host
  che possono risentire di una collisione generata da due postazione
  arbitraria
\item
  \textbf{Dominio di "broadcast": }insieme degli host che ricevono i
  pacchetti in broadcast
\end{itemize}

\emph{figura 5-82}

\hypertarget{header-n296}{%
\subsection{LAN virtuali}\label{header-n296}}

Consente di definire più reti locali virtuali distinte utilizzando una
stessa infrastruttura fisica. Ogni VLAN si comporta come una rete locale
separata dalle altre:

\begin{itemize}
\item
  i pacchetti di broadcast sono confinati all'interno della VLAN
\item
  la comunicazione a livello 2 è confinata all'interno della VLAN
\item
  l'interconnessione tra più VLAN viene effettuata a livello 3 (è
  necessario un router)
\end{itemize}

Le VLAN sono definite nello standard 802.1q e nel 802.1d che riguarda la
comunicazione tra diversi standard 802 attraverso bridge.

\hypertarget{header-n306}{%
\subsubsection{Scopo delle VLAN}\label{header-n306}}

L'utilizzo delle VLAN consente:

\begin{itemize}
\item
  \textbf{risparmio: }definire più topologie virtuali utilizzando la
  stessa infrastruttura fisica. Non sono necessarie modifiche
  all'hardware
\item
  \textbf{aumento di prestazioni: }costruire la rete in base alle
  esigenze del momento, limitando il traffico broadcast
\item
  \textbf{aumento della sicurezza: }suddividere il traffico e isolarlo
  nelle varie VLAN
\item
  \textbf{flessibilità: }più facile spostare un utente dal punto di
  vista logico da una VLAN a un'altra
\end{itemize}

\hypertarget{header-n317}{%
\subsubsection{Requisiti sui bridge}\label{header-n317}}

Per implementare le VLAN è necessario che gli apparati supportino lo
standard 802.1q, con il quale è possibile definire due tipologie di
VLAN: la port based (privata) e quella tagged (802.1q).

Bisogna anche istruire bridge e switch perché riconoscano le VLAN, non è
possibile farlo in autoapprendimento ma vanno preprogrammati.

\hypertarget{header-n320}{%
\subsubsection{Funzioni del bridge in 802.1q}\label{header-n320}}

Per supportare le VLAN è necessario che i bridge svolgano le seguenti
tre funzioni:

\begin{itemize}
\item
  \textbf{ingress: }l'apparato deve capire a quale VLAN appartiene il
  frame in ingresso
\item
  \textbf{forwarding: }effettuare l'inoltro in base alla VLAN di
  appartenenza
\item
  \textbf{egress: }deve poter trasmettere il frame in uscita in modo che
  la VLAN sia interpretabile dagli altri bridge
\end{itemize}

\hypertarget{header-n329}{%
\subsubsection{Port based VLAN (untagged)}\label{header-n329}}

Tecnica abbastanza semplice, assegna in maniera statica ciascuna porta
del bridge a una VLAN definita con la configurazione del bridge.
Permette di costruire su un singolo apparato 2 o più bridge logici.

Le funzioni del bridge sono semplici:

\begin{itemize}
\item
  ingress: un frame in ingresso appartiene alla VLAN a cui è assegnata
  la porta
\item
  forwarding: frame inoltrato solamente verso le porte appartenenti alla
  stessa VLAN (forwarding database distinto per ogni VLAN)
\item
  egress: determinata la porta il frame viene trasmesso così com'è
\end{itemize}

Le VLAN untagged non richiedono di modificare i pacchetti Ethernet
poiché viene definito tutto sul bridge (che dev'essere compatibile con
lo standard 802.1q).

\hypertarget{header-n340}{%
\subsubsection{VLAN 802.1q (tagged)}\label{header-n340}}

Con questo standard è possibile far condividere lo stesso link fisico da
tra VLAN differenti, stampando nel pacchetto di livello 2 la VLAN di
appartenenza, aggiungendo nel frame Ethernet 4 byte che trasportano le
informazioni sulla VLAN. Questo identificativo (VLAN tag) deve essere
uguale per tutti i bridge che saranno tutti programmati per riconoscere
tale VLAN.

\hypertarget{header-n342}{%
\paragraph{Frame Ethernet 802.1q}\label{header-n342}}

Vengono aggiunti 4 byte dopo gli indirizzi di sorgente e destinazione i
quali conterranno il \textbf{TPI} (Tag Protocol Identifier), che
specifica che il frame è aderente a 802.1q e il \textbf{TCI} (Tag
Control Information) che trasporta informazioni relative alla VLAN
(priorità, interconnessione e VLAN tag).

\emph{figura 7-90}

\hypertarget{header-n345}{%
\paragraph{Considerazioni sul frame}\label{header-n345}}

La modifica proposta da 802.1q richiede anche la modifica della
dimensione massima di 1518 bytes con l'aggiunta di due byte. Inoltre, il
campo TPI ha un valore non utilizzato come "protocol type" nei frame
Ethernet ordinari in modo che sia riconoscibile che il frame è di tipo
802.1q ma una scheda non compatibile non scarti il frame.

\hypertarget{header-n347}{%
\subsubsection{VLAN con switch/router}\label{header-n347}}

Molti produttori consentono di implementare all'interno degli switch
funzionalità di routing (livello 3), permettendo di interconnettere tra
loro diverse VLAN mantenendo la separazione dei domini di broadcast.

Utilizzando porte configurate in modalità TRUNK è possibile trasportare
su un unico cavo i dati di più VLAN, lasciando agli switch il compito di
inoltrare i pacchetti correttamente.

\hypertarget{header-n350}{%
\subsubsection{Porte tagged e untagged}\label{header-n350}}

Negli switch 802.1q tutte le porte devono essere associate a una o più
VLAN: se la porta è associata a una VLAN untagged i frame ricevuti non
trasporteranno tag ne lo trasporteranno i frame in uscita, il link su
tali porte si chiama \emph{access link}.

Se la porta è in modalità tagged, il link si chiamerà \emph{trunk link}
e la VLAN di appartenenza del frame sarà definita dal valore nel tag.

\hypertarget{header-n353}{%
\subsubsection{Protocol based VLAN}\label{header-n353}}

Esiste la possibilità di assegnare un frame a una VLAN in maniera
dinamica, sulla base di diversi parametri opportunamente configurati
negli apparati (richiesti particolarmente evoluti).

Viene effettuato il packet filtering in base a delle regole come IP del
mittente, protocol type, indirizzo Ethernet. Può essere definita una
associazione statica, che avrà precedenza sulle altre regole.

Alcuni protocolli proprietari consentono di configurare le regole
dinamiche in maniera centrale, importando le configurazioni tramite la
rete, ad esempio non consentire l'accesso alle VLAN se il MAC address
dell'host non è stato registrato dall'amministratore di rete.

\hypertarget{header-n357}{%
\subsubsection{Default VLAN}\label{header-n357}}

Gli switch 802.1q sono preconfigurati con una default VLAN assegnata col
tag 1 e tutte le porte assegnate ad essa in modalità untagged,
permettendo al primo accesso di tale switch un funzionamento
tradizionale. Per poter modificare il VLAN ID associato a ciascuna porta
bisognerà eliminare tale porta dalla VLAN per poi poterla riconfigurare.

\hypertarget{header-n359}{%
\subsubsection{VLAN di management}\label{header-n359}}

In ogni rete IP sono presenti due piani di funzionalità: un piano dati
(o data plane) dove circola il traffico degli utenti e un piano di
controllo (control plane) relativo al traffico di controllo della rete
(BGP).

Usando le VLAN è interessante poter creare una VLAN relativa alla
gestione della rete, utilizzando la stessa infrastruttura.

\hypertarget{header-n362}{%
\subsection{ATM e MPLS}\label{header-n362}}

ATM e MPLS sono delle soluzioni che consentono di generare circuiti
virtuali utilizzando l'approccio a commutazione di pacchetto, consentono
l'allocazione delle risorse per i flussi e la gestione della qualità del
servizio. Queste tecnologie possono essere integrate nelle reti IP,
attualmente sono utilizzate per interconnettere alcune zone della rete e
non sono visibili all'utente.

\hypertarget{header-n364}{%
\subsubsection{Trasferimento asincrono (ATM)}\label{header-n364}}

ATM è nato verso metà anni 80 con l'obbiettivo di estendere la
tecnologia delle reti telefoniche in modo tale da essere utilizzata per
reti dati, progettando reti in grado di trasportare file audio e video
in tempo reale e supportare file di testo e immagini. Può essere
considerata la tecnologia telefonica di ultima generazione e viene
tutt'ora usata nelle reti ADSL, consente la realizzazione di circuiti
virtuali e l'implementazione del QoS.

\hypertarget{header-n366}{%
\paragraph{Architettura ATM}\label{header-n366}}

\emph{figura 5-102}

Segue la struttura TCP/IP, i terminali hanno 3 livelli mentre gli switch
2. I 3 livelli sono:

\begin{itemize}
\item
  \textbf{AAL (ATM adaptation layer): }presente nei dispositivi alla
  periferia della rete, svolge una funzione analoga al livello di
  Trasporto quindi segmentazione e riassemblaggio dei pacchetti e
  mappatura dei flussi nelle tipologie di QoS
\item
  \textbf{ATM:} fulcro dell'architettura, considerato "livello di rete",
  definisce la struttura della cella ATM (pacchetto) e tutti i suoi
  campi
\item
  \textbf{Livello fisico}
\end{itemize}

La rete ATM inizialmente era concepita come una rete stand-alone, che
fosse in grado di trasportare dati "da una scrivania a un'altra",
venendo considerata una tecnologia di rete. Dopo l'affermazione dello
stack TCP/IP come standard di Internet, per far si che ATM si potesse
integrare nella reti IP venne messo al suo di sotto, diventando un
livello 2, o livello di link commutato (utilizzato solo in alcune reti
al di sotto di IP).

\emph{figura 5-103}

\hypertarget{header-n378}{%
\subparagraph{AAL: ATM Adaptation Layer}\label{header-n378}}

Presente solo negli host terminali, adatta i livelli superiori al
livello ATM sottostante frammentandoli adeguatamente (come nella
segmentazione TCP in pacchetti IP).

Ci sono diverse tipologie (o profili) AAL che suggeriscono il QoS
richiesto:

\begin{itemize}
\item
  \textbf{AAL1:} servizio a tasso costante, CBR, come nei servizi
  tradizionali per il traffico telefonico
\item
  \textbf{AAL2:} servizio a tasso variabile, VBR, adeguato per la
  trasmissione di video MPEG
\item
  \textbf{AAL5:} servizio dati (datagram IP)
\end{itemize}

\hypertarget{header-n388}{%
\subparagraph{Livello ATM}\label{header-n388}}

Offre il servizio di trasporto di celle attraverso la rete ATM, analogo
al livello di rete IP con servizi però molto differenti.

\emph{tabella 5-106}

Il livello ATM implementa una rete a pacchetto a circuiti virtuali,
chiamati \textbf{canali virtuali (VC)}, percorsi con un collegamento
diretto fra sorgente e destinazione. Ciascun pacchetto viene marchiato
con l'indicatore \textbf{VCI} cosicché ogni switch saprà come
interpretarlo. Al canale virtuale possono essere riservate
\textbf{risorse dedicate} come banda e buffer per garantire il QoS.

I canali virtuali possono essere \textbf{permanenti}, per connessioni di
lunga durata, utilizzati tra zone della rete lontane tra loro, oppure
\textbf{dinamici} (creati su richiesta).

L'utilizzo di canali virtuali ha il vantaggio di poter controllare
prestazioni e QoS, controllando la congestione della rete, ma ha gli
svantaggi che potrebbe non esserci un profilo per ogni tipologia di dato
da trasportare, e avendo un numero limitato di canali virtuali non è del
tutto scalabile. Nel caso di connessioni di breve durata la costruzione
del VC richiede tempo riducendo le prestazioni percepite.

La \textbf{cella ATM} è costituita da un'intestazione da 5 byte (VCI,
Payload Type, Priorità sulla perdita di cella, e byte di controllo
errore) e un carico utile da 48 byte, che consente una trasmissione più
efficace avendo dimensione fissa e un ritardo minore data la piccola
dimensione.

\hypertarget{header-n395}{%
\subparagraph{Livello fisico ATM}\label{header-n395}}

Suddiviso in due parti:

\begin{itemize}
\item
  \textbf{Transmission Convergence Sublayer (TCS):} adatta la cella
  creata da ATM al livello fisico sottostante creando il checksum,
  effettuando la delineazione della cella e consentendo la
  strutturazione del canale trasmettendo celle inattive se non ci sono
  dati da inviare (il canale fisico è come un nastro trasportatore di
  celle)
\item
  \textbf{Physical Medium Dependent:} dipende dal mezzo fisico
  utilizzato
\end{itemize}

Il livello fisico di ATM consente il funzionamento con diverse
tecnologie come SONET/SDH (reti ottiche sincrone), T1/T3 (fibra,
microonde e cavo) o mappare le celle ATM su canali non strutturati
(grazie alla delineazione di ATM).

\hypertarget{header-n403}{%
\paragraph{IP su ATM}\label{header-n403}}

\emph{figura 5-114}

Quando si raggiunge un router di ingresso nella rete ATM, questo router
dovrà determinare come trasferire il datagramma attraverso la rete,
utilizzando l'indirizzo IP di destinazione per capire dove inoltrarlo e
utilizzando ARP per chiedere alla rete ATM di costruire il percorso
relativo.

Raggiunto il router di uscita verrà risalita la pila protocollare,
incontrando AAL5 che permetterà di ricostruire il datagramma IP e di
conseguenza la risalita del pacchetto IP e sua consegna.

Per il corretto funzionamento di IP su ATM sarà necessario quindi un
protocollo ARP dedicato ad ATM e dei router dotati di uno stack
protocollare e un'interfaccia di rete compatibili con lo standard.

\hypertarget{header-n408}{%
\subsubsection{Multiprotocol Label Switching (MPLS)}\label{header-n408}}

L'obiettivo iniziale del protocollo MPLS è quello di velocizzarre
l'inoltro IP usando un'etichetta di lunghezza stabilita (invece
dell'indirizzo di destinazione IP).

\begin{figure}
\centering
\includegraphics{/Users/davideparpinello/Library/Application Support/typora-user-images/Schermata 2020-05-24 alle 13.31.20.png}
\caption{}
\end{figure}

I router a commutazione di etichetta (router MPLS) inviano i pacchetti
analizzando l'etichetta MPLS nella tabella d'instradamento, passando il
datagramma all'interfaccia corretta. Viene utilizzato un particolare
protocollo RSVP-TE (estensione di RSVP) per distribuire etichette tra
router, che consente l'invio di pacchetti lungo reti non utilizzabili
con IP standard. Questi router coesistono coi router "solo-IP".

\end{document}
